\documentclass{article}
\usepackage[paper=a4paper,margin=2.5cm]{geometry}
\usepackage[ngerman]{babel}
\usepackage[utf8]{inputenc}
\usepackage{graphicx}
\usepackage[labelfont=bf]{caption}
\usepackage{float}
\usepackage{blindtext}
\usepackage{multicol}
\setlength{\parindent}{0cm}

\begin{document}

\title{P3HT in o-Xylol bei 80K 03}
\author{Till Biskup, Jara Popp}
\date{10.12.2018}
\maketitle

\section*{Übersicht}

\begin{tabbing}
	\hspace{3.5cm} \= \kill
			Label \> P3HT in o-Xylol bei 80K 03
		\\
				Start \> 2014-11-20 16:02:00
		\\
				End \> 2014-11-20 13:57:00
		\\
				Purpose \> Breiter Scan
		\\
				Operator \> Till Biskup, Jara Popp
		\\
			\end{tabbing}

\begin{figure}[H]
	\includegraphics[width=\textwidth]{Plotter2D}
	\caption{\textbf{TREPR-Signal von P3HT in o-Xylol bei 80.0~K.} 2000~spp; MW: 20.0~dB, 2.0~mW; VAmp: 42.0~dB, 25.0~MHz; Laser: 566.0~nm (OPO Pos. 2366), 10.0~Hz, 1.0~mJ pro Puls.} 
\end{figure}

\clearpage
\section*{Experimentelle Parameter}

\begin{minipage}{\textwidth}
\begin{tabbing}
\hspace{3.5cm} \= \kill
\textbf{Sample}\\
			Description \> P3HT in o-Xylol
		\\
				Solvent \> N/A
		\\
				Preparation \> Probe an Innenwand des EPR-Roehrchens als Film abgeschieden, entgast und abgeschmolzen
		\\
				Tube \> 3.8 x 3.6 x 200 mm
		\\
				Name \> P3HT in o-Xylol
		\\
				Id \> 12
		\\
			\end{tabbing}
\end{minipage}
	
\begin{multicols}{2}

	\begin{minipage}{.49\textwidth}
	\textbf{Temperature control}\\
						Cryostat: Oxford 935
			\\
								Cryogen: N2
			\\
								Temperature: 80.0~K
			\\
								Controller: Oxford ITC 503
			\\
				\end{minipage}
	\begin{minipage}{.49\textwidth}
	\textbf{Transient}\\
						Points: 5000
			\\
								Length: 10.0~us
			\\
								Trigger position: 500
			\\
				\end{minipage}
	\begin{minipage}{.49\textwidth}
	\textbf{Experiment}\\
						Runs: 1
			\\
								Shot repetition rate: 10.0~Hz
			\\
				\end{minipage}
	\begin{minipage}{.49\textwidth}
	\textbf{Spectrometer}\\
						Model: ESP380E
			\\
								Software: Transient, Vers. 0.6
			\\
				\end{minipage}
	\begin{minipage}{.49\textwidth}
	\textbf{Magnetic field}\\
																Start: 280.0~mT
			\\
								Stop: 410.0~mT
			\\
								Step: 0.4~mT
			\\
								Sequence: outward
			\\
								Controller: Bruker ER 032M
			\\
								Power supply: Bruker ER 083 CS
			\\
				\end{minipage}
	\begin{minipage}{.49\textwidth}
	\textbf{Background}\\
						Field: 260.0~mT
			\\
								Occurrence: 10
			\\
								Polarisation: absorptive
			\\
								Intensity: 0.0~
			\\
				\end{minipage}
	\begin{minipage}{.49\textwidth}
	\textbf{Bridge}\\
						Model: Bruker ER 046 MRT
			\\
													Attenuation: 20.0~dB
			\\
								Power: 2.0~mW
			\\
								Detection: mixer
			\\
								Frequency counter: HP 5352B
			\\
								Mw frequency: 9.70294~GHz
			\\
				\end{minipage}
	\begin{minipage}{.49\textwidth}
	\textbf{Video amplifier}\\
						Bandwidth: 25.0~MHz
			\\
								Amplification: 42.0~dB
			\\
				\end{minipage}
	\begin{minipage}{.49\textwidth}
	\textbf{Recorder}\\
						Model: LeCroy 9354A
			\\
								Averages: 2000
			\\
								Time base: 2.0~ns
			\\
								Bandwidth: 500.0~MHz
			\\
								Pretrigger: 1.0~us
			\\
								Coupling: DC
			\\
								Impedance: 50.0~Ohm
			\\
								Sensitivity: 20.0~mV
			\\
				\end{minipage}
	\begin{minipage}{.49\textwidth}
	\textbf{Probehead}\\
						Type: dielectric
			\\
								Model: Bruker ER 4118X-MD5
			\\
								Coupling: critical
			\\
				\end{minipage}
	\begin{minipage}{.49\textwidth}
	\textbf{Pump}\\
						Type: Laser
			\\
								Model: GCR 190-10
			\\
								Wavelength: 566.0~nm
			\\
								Power: 1.0~mJ
			\\
								Repetition rate: 10.0~Hz
			\\
								Tunable type: OPO
			\\
								Tunable model: OPTA BBO VIS/IR
			\\
													Tunable position: 2366
			\\
									\end{minipage}
\end{multicols}





\begin{figure}[H]
	\includegraphics[width=\textwidth]{Plotter1D}
\caption{\textbf{TREPR-Signal von P3HT in o-Xylol bei 80.0~K: Schnitt bei 5e-07 (gemittelt über ).} 2000~spp; MW: 20.0~dB, 2.0~mW; VAmp: 42.0~dB, 25.0~MHz; Laser: 566.0~nm (OPO Pos. 2366), 10.0~Hz, 1.0~mJ pro Puls.}  
\end{figure}

\section*{Prozessierung}

	\textbf{Pretrigger offset compensation:}
	\\
			Zeropoint index: 500
		\\
		\textbf{Averaging:}
	\\
			Dimension: 0
		\\
			Range: [4.8e-07, 5.2e-07]
		\\
			Unit: axis
		\\
	
\end{document}